\documentclass{article}

\usepackage[utf8]{inputenc}
\usepackage[T1]{fontenc}

\usepackage{geometry}
 \geometry{
 a4paper,
 total={170mm,257mm},
 left=20mm,
 top=20mm,
 }

\setlength{\parskip}{7pt}
\setlength{\parindent}{0pt}

\title{Using the Diligent JTAG-HS2 Programming Cable with OpenOCD}
\date{2019-10-11}
\author{Armand Bouvier-Neveu -- Thales Research and technology France}

\PassOptionsToPackage{hyphens}{url}\usepackage{hyperref}
\usepackage{listings}
\usepackage{amssymb}
\usepackage{graphicx}
\graphicspath{ {./images/} }
\usepackage{float}

\lstset{
	basicstyle=\ttfamily,
	breaklines=true,
	columns=fullflexible,
	tabsize=4,
	showstringspaces=false,
	frame=single
}

\begin{document}
	\maketitle
	
	\section{About}
	
	The JTAG-HS2 Programming Cable is an USB JTAG probe. It is available here: \url{https://store.digilentinc.com/jtag-hs2-programming-cable/}
	
	\begin{figure}[H]
   	\centering
   	\includegraphics[width=0.6\textwidth]{hs2.png}
	\end{figure}
	
	The cable is fully compatible with all Xilinx tools, but it can also be used with OpenOCD.
	
	\section{Software configuration}
	
	\subsection{Libusb}
	
	You need libusb to use the cable.
	
	On Ubuntu, you can use apt:
	
	\begin{lstlisting}[language=bash]
    $ sudo apt install libusb-1.0-0-dev
    \end{lstlisting}
    
    You can also build it yourself using the sources available on \url{https://libusb.info}
    
    \subsection{Libftdi}
    
    The cable contains an FTDI chip, so you need libftdi.
    
    Ubuntu:
    
    \begin{lstlisting}[language=bash]
    $ sudo apt install libftdi1-dev
    \end{lstlisting}
    
    Or from sources: \url{https://www.intra2net.com/en/developer/libftdi/download.php}
    
    \subsection{OpenOCD Compilation}
    
    The OpenOCD interface driver used with the cable is \textit{ftdi}.
    
    To enable this interface, use the \textit{-{}-enable-ftdi} option when compiling OpenOCD.
    
    \begin{lstlisting}[language=bash]
    $ ./configure --enable-ftdi <other options>
    \end{lstlisting}
    
    To check if your existing instance of OpenOCD supports the ftdi interface, you can list the supported debug adapter drivers:
    
    \begin{lstlisting}[language=bash]
    $ openocd -c interface_list
    \end{lstlisting}
    
    \subsection{udev rules}
    
    It is necessary to add a udev rule to use the cable.
    
    OpenOCD provides a file containing the rule we need. Copy it into /etc/udev/rules.d/
    
    \begin{lstlisting}[language=bash]
    $ sudo cp <openocd>/contrib/60-openocd.rules /etc/udev/rules.d
    \end{lstlisting}
    
    The file is also available here: \url{https://github.com/riscv/riscv-openocd/blob/riscv/contrib/60-openocd.rules}
    
    The particular entry about the HS2 cable is :
    
    \begin{lstlisting}
    ATTRS{idVendor}=="0403", ATTRS{idProduct}=="6014", MODE="660", GROUP="plugdev", TAG+="uaccess"
    \end{lstlisting}
    
    Then either reboot your system or reload the udev configuration with :
    
    \begin{lstlisting}[language=bash]
    $ sudo udevadm control --reload
    \end{lstlisting}
    
    \subsection{Checking}
    
    To check if the cable is recognized, run lsusb. There should be a line like this:
    
    \begin{lstlisting}[language=bash]
    $ lsusb
    [...]
    Bus 005 Device 003: ID 0403:6014 Future Technology Devices International, Ltd FT232H Single HS USB-UART/FIFO IC
    [...]
    \end{lstlisting}
    
    For more details about the device, use more options:
    
    \begin{lstlisting}[language=bash]
    $ lsusb -v -d 0403:6014
    \end{lstlisting}
    
    \newpage
    \section{OpenOCD Config File}
    
    Examples of config files using the HS2 cable can be found on the Internet:
    
    \vspace{-\topsep}
	\begin{itemize}
	\item \url{https://github.com/pulp-platform/pulpissimo/blob/master/fpga/pulpissimo-zedboard/openocd-zedboard-hs2.cfg}
	\item \url{https://github.com/arduino/OpenOCD/blob/master/tcl/interface/ftdi/digilent-hs2.cfg}
	\end{itemize}
	
	Generally, for the interface part of the config file, you can use:
	
	\begin{lstlisting}
interface ftdi
ftdi_device_desc "Digilent USB Device"
ftdi_vid_pid 0x0403 0x6014
ftdi_serial 210249A85F9B
ftdi_channel 0						
ftdi_layout_init 0x00e8 0x60eb

reset_config none
    \end{lstlisting}
    
    Use \textbf{lsusb -v -d 0403:6014} to find the necessary values :
    
    \vspace{-\topsep}
	\begin{itemize}
	\item \textbf{ftdi\_device\_desc}: this command should use the same value as the \textbf{iProduct} field
	
	\item \textbf{ftdi\_vid\_pid}: \textbf{idVendor} and \textbf{idProduct}
	
	\item \textbf{ftdi\_serial}: this command is useful if there are more than one adapter connected to the host and can be omitted otherwise. The needed value is in the \textbf{iSerial} field.
	
	\item \textbf{ftdi\_channel}: the default (channel 0) is fine
	
	\item \textbf{ftdi\_layout\_init}: initial values of the FTDI GPIO data and direction registers. Refer to the adapter schematics to find them. 0x00e8 and 0x60eb work for the HS2 cable
	\end{itemize}
	
	For more information about the OpenOCD commands, see \url{http://www.openocd.org/doc/html/Debug-Adapter-Configuration.html}
	
	\newpage
	\section{Verification}
	
	To check if the set up is successful, you can use the following procedure:
	
	\vspace{-\topsep}
	\begin{itemize}
	\item Plug the cable in an USB port of your workstation
	\item Do not plug the JTAG end of the cable; leave it floating
	\item Run openocd with this config file (change the serial number):
	
	\begin{lstlisting}[language=bash]
    $ openocd -f test.cfg
    \end{lstlisting}
    
    \begin{lstlisting}
# Digilent JTAG-HS2

interface ftdi
ftdi_device_desc "Digilent USB Device"
ftdi_vid_pid 0x0403 0x6014
ftdi_serial 210249A85F9B
ftdi_channel 0						
ftdi_layout_init 0x00e8 0x60eb

reset_config none

adapter_khz     1000


set _CHIPNAME riscv
jtag newtap $_CHIPNAME cpu -irlen 5 

set _TARGETNAME $_CHIPNAME.cpu
target create $_TARGETNAME riscv -chain-position $_TARGETNAME

init
halt
echo "Ready for Remote Connections"
    \end{lstlisting}
    
    \item If the set up is succesful, you should see something like this:
    
    \begin{lstlisting}[language=bash]
Open On-Chip Debugger 0.10.0+dev-00630-g30b93b8-dirty (2019-10-02-14:34)
Licensed under GNU GPL v2
For bug reports, read
	http://openocd.org/doc/doxygen/bugs.html
none separate
adapter speed: 1000 kHz
Info : auto-selecting first available session transport "jtag". To override use 'transport select <transport>'.
Info : clock speed 1000 kHz
Error: JTAG scan chain interrogation failed: all zeroes
Error: Check JTAG interface, timings, target power, etc.
Error: Trying to use configured scan chain anyway...
Error: riscv.cpu: IR capture error; saw 0x00 not 0x01
Warn : Bypassing JTAG setup events due to errors
Error: dtmcontrol is 0. Check JTAG connectivity/board power.
Info : Listening on port 3333 for gdb connections
Error: Target not examined yet
    \end{lstlisting}
	
	\end{itemize}
	
	\newpage
	\section{Hardware configuration}
	
	The HS2 cable does not use the TRST signal. 
	
	Do not leave the pin floating and tie it to 1 if the target TAP reset is active low, or 0 if it is active high.
	
	\section{Vivado Hardware Manager}
	
	The Digilent JTAG-HS2 cable is supported by the Xilinx Hardware Server. For some reason, it is impossible to use the cable with OpenOCD while the Hardware Server is active.
	
	This makes using Vivado and OpenOCD at the same time a problem.
	
	A possible fix is the following protocol:
	
	\vspace{-\topsep}
	\begin{itemize}
	\item Open the hardware server in Vivado and connect to your FPGA board
	\item Upload the bitstream and program the device
	\item \textbf{Close} the hardware server
	\item Launch OpenOCD
	\end{itemize}
	
	Of course, This closes the hardware server and removes access to Vivado features.
	
	To properly fix the issue, you would need to find a way of preventing the Vivado Hardware Server from occupying the USB device of the cable.
	
\end{document}
