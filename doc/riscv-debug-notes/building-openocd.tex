\documentclass{article}

\usepackage[utf8]{inputenc}
\usepackage[T1]{fontenc}

\usepackage{geometry}
 \geometry{
 a4paper,
 total={170mm,257mm},
 left=20mm,
 top=20mm,
 }

\setlength{\parskip}{7pt}
\setlength{\parindent}{0pt}

\title{Building riscv-openocd}
\date{2019-09-12}
\author{Armand Bouvier-Neveu -- Thales Research and technology France}

\usepackage{hyperref}
\usepackage{listings}
\usepackage{amssymb}

\lstset{
	basicstyle=\ttfamily,
	breaklines=true,
	columns=fullflexible,
	tabsize=4,
	showstringspaces=false,
	frame=single
}

\begin{document}
	\maketitle
    
    \section{About}
    
    OpenOCD is a free and open-source software distributed under the GPL-2.0 license. It provides on-chip programming and debugging support with a layered architecture of JTAG interface and TAP support.
    
    A fork with RISC-V support has been developped and is available on GitHub at \url{https://github.com/riscv/riscv-openocd}
    
    This guide aims at providing help about building OpenOCD for various plateforms. More details, guides and manuals about OpenOCD can be found in the repository's README.md
    
    \section{Dependencies}
    
    \subsection{Compiler}
    
    GCC or Clang is currently required to build OpenOCD. On a Linux workstation, GCC should be available by default or in the distribution's repositories.
    
    For other architectures, cross-compiling is possible. Linaro provides pre-built GNU cross-toolchain binary archives at \url{https://releases.linaro.org/components/toolchain/binaries/}
    
    The \textit{uname} command can help finding the architecture of a target:
    
    \begin{lstlisting}[language=bash]
    $ uname --m
    \end{lstlisting}
    
    \subsection{Tools}
    
    Other tools are also needed:
    \begin{itemize}
    	\itemsep0em
    	\item make
    	\item libtool
    	\item pkg-config $\geqslant$ 0.23
    	\item autoconf $\geqslant$ 2.64
    	\item automake $\geqslant$ 1.14
    	\item texinfo
    \end{itemize}
    
    On Ubuntu, ensure that everything is installed with:
    \begin{lstlisting}[language=bash]
    $ sudo apt install make libtool pkg-config autoconf automake texinfo
    \end{lstlisting}
    
    \subsection{Libraries}
    
    Depending on the compilation options, some libraries might be needed.
    
    \subsubsection{libusb}
    
    If the target is the workstation (i.e. not cross-compiling), it is likely that libusb is already installed or included in the distribution's repositories. For Ubuntu:
    \begin{lstlisting}[language=bash]
    $ sudo apt install libusb-1.0-0-dev
    \end{lstlisting}
    
    To cross-compile, or to use the latest version from source:
    \begin{itemize}
    	\itemsep0em
    	\item Obtain the source. The latest tarball is available on \url{https://libusb.info}
    	\item Extract the source
    	
    \begin{lstlisting}[language=bash]
    $ tar -zxvf libusb-X.X.XX.tar.bz2
    \end{lstlisting}
    
    	\item Make a build directory somewhere, inside the libusb root directory for example
    	
    \begin{lstlisting}[language=bash]
    $ cd libusb-X.X.XX
    $ mkdir build
    \end{lstlisting}
    
    	\item If cross-compiling, set the compiler. Ensure that the given compiler is in the \$PATH
    	
    \begin{lstlisting}[language=bash]
    $ export CC=arm-linux-gnueabihf-gcc
    \end{lstlisting}
    
    	\item Use the \textit{configure} script. You can see all the options with:
    	
    \begin{lstlisting}[language=bash]
    $ ./configure --help
    \end{lstlisting}
    
    	\item For example, these options set the install directory, set the host for cross-compiling and disable udev:
    
    \begin{lstlisting}[language=bash]
    $ ./configure --prefix=<build dir> --exec-prefix=<build dir> --host=arm-linux-gnueabihf --disable-udev
    \end{lstlisting}
    
    \item Compile and install files
    
    \begin{lstlisting}[language=bash]
    $ make
    $ make install
    \end{lstlisting}
    
    \item The produced library files are in the \textit{build} directory. They need to be added to the compiler to be used. 
    
    \end{itemize}
    
    \subsubsection{libftdi}
    
    If the target is the workstation (i.e. not cross-compiling), it is likely that libusb is already installed or included in the distribution's repositories. For Ubuntu:
    \begin{lstlisting}[language=bash]
    $ sudo apt install libftdi1-dev
    \end{lstlisting}
    
	To cross-compile or to get the latest version, the source is available at \url{https://www.intra2net.com/en/developer/libftdi/download.php}
    
    \begin{itemize}
    
	\item Some tools are needed. On Ubuntu:
	
	\begin{lstlisting}[language=bash]
    $ sudo apt install build-essential cmake
    \end{lstlisting}
    
    \item Libusb is also required.
    
    \item Download and extract the tarball
    
    \item Inside the directory, make a build directory
    
    \begin{lstlisting}[language=bash]
    $ cd libftdiX-X.X
    $ mkdir build
    $ cd build
    \end{lstlisting}
    
    \item Create a cmake toolchain file and fill it with the target informations
    
    \begin{lstlisting}[language=bash, title=toolchain-arm-linux-gnueabihf.cmake]
    SET(CMAKE_SYSTEM_NAME arm-linux-gnueabihf)
    SET(CMAKE_C_COMPILER arm-linux-gnueabihf-gcc)
    SET(CMAKE_CXX_COMPILER arm-linux-gnueabihf-g++)
    SET(LIBUSB_LIBRARIES <...>/gcc-linaro-7.4.1-2019.02-x86_64_arm-linux-gnueabihf/lib/libusb-1.0.so.0)
    SET(LIBUSB_INCLUDE_DIR <...>/gcc-linaro-7.4.1-2019.02-x86_64_arm-linux-gnueabihf/include/libusb-1.0)
    \end{lstlisting}
    
    \item Make sure the compilers are in the \$PATH and use cmake. The install directory can be controlled with the prefix options.
    
    \begin{lstlisting}[language=bash]
    $ cmake -DCMAKE_TOOLCHAIN_FILE=<cmake toolchain file> -DCMAKE_INSTALL_PREFIX=$(pwd) -DLIBFTDI_LIBRARY_DIRS=$(pwd) ..
    \end{lstlisting}
    
    \item Compile and install the files
    
    \begin{lstlisting}[language=bash]
    $ make
    $ make install
    \end{lstlisting}
    
    \item The produced library files are in the \textit{build} directory. They needed to be added to the compiler to be used. 
    
    \end{itemize}
    
    \section{OpenOCD}
    
    \begin{itemize}
    
    \item Download the sources
    
    \begin{lstlisting}[language=bash]
    $ git clone https://github.com/riscv/riscv-openocd
    $ cd riscv-openocd
    \end{lstlisting}
    
    \item Prepare a \textit{build} directory
    
    \begin{lstlisting}[language=bash]
    $ mkdir build
    \end{lstlisting}
    
    \item If the aim is to use OpenOCD for PULP applications, a small patch is required. Find src/target/riscv/riscv.h and modify the following definition :
    
    \begin{lstlisting}[language=C]
    -#define RISCV_MAX_HARTS 32
    +#define RISCV_MAX_HARTS 1024
    \end{lstlisting}
    
    \item If cross-compiling, set the compiler. Ensure that the given compiler is in the \$PATH
    	
    \begin{lstlisting}[language=bash]
    $ export CC=arm-linux-gnueabihf-gcc
    \end{lstlisting}
    
    \item Launch the bbotstrap script
    	
    \begin{lstlisting}[language=bash]
    $ ./bootstrap
    \end{lstlisting}
    
    \item Review the possible configuration options
    	
    \begin{lstlisting}[language=bash]
    $ ./configure --help
    \end{lstlisting}
    
    \item In particular, \textit{--enable-sysfsgpio}, \textit{--enable-ftdi} and \textit{--enable-remote-bitbang} must be added to use the corresponding features.
    
    \item The \textit{--host=} option must be provided to cross-compile
    
    \item The \textit{--prefix=} and \textit{--exec-prefix=} options control the installation directory
    
    \item Use configure with all the chosen options
    
    \begin{lstlisting}[language=bash]
    $ ./configure --enable-sysfsgpio --host=arm-linux-gnueabihf --prefix=<build dir> --exec-prefix=<build dir>
    \end{lstlisting}
    
    \item Some problems may be encountered with an aarch64-linux-gnu host. Please review \url{https://github.com/riscv/riscv-openocd/issues/17}
    
    \item Compile and install files
    
    \begin{lstlisting}[language=bash]
    $ make
    $ make install
    \end{lstlisting}
    
    \end{itemize}
    
    A single \textit{openocd} binary should be available in \textit{build}
    
\end{document}
